%-----Cap�tulo 5 - Conclus�o
\chapter{Conclus�o}
\label{cap5}

O trabalho proposto teve como objetivo realizar um estudo do desempenho de um algoritmo de segmenta��o de v�deos em imagens de alta resolu��o (4k) e correlacionar os resultados obtidos com o custo computacional necess�rio para o processamento destas imagens. Atrav�s de ferramentas de monitoramento, foi constatado que quando se alcan�ou uma resolu��o intermedi�ria, no caso a de 1152 x 648, o ganho de acur�cia obtido foi m�nimo e n�o justificava o custo computacional necess�rio que teria de ser investido para processar as imagens. 

Atrav�s destes resultados, verifica-se tamb�m que, dentre as vari�veis observadas, a mem�ria RAM do computador � a que se mostrou mais requisitada para a execu��o do algoritmo. Tamb�m verificou-se a efic�cia do algoritmo adotado: SubSENSE, que obteve uma acur�cia de 79 \% no \textit{dataset} Cornelio4k. 

Assim, espera-se que o desenvolvimento deste estudo, juntamente com seus resultados sirvam de refer�ncia na �rea de processamento de imagens no que se refere � segmenta��o de imagens de alta resolu��o. Alguns trabalhos futuros que podem ser abordados s�o: expans�o do \textit{dataset}, tendo em vista que com um n�mero maior de imagens seria poss�vel obter-se resultados mais precisos; desenvolver um novo estudo utilizando outros algoritmos de segmenta��o de modo a observar em qual resolu��o h� maior ganho.
	